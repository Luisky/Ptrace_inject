\documentclass[12pt]{article}

\usepackage{lmodern}
\usepackage[utf8]{inputenc}
\usepackage[T1]{fontenc}
\usepackage{hyperref}

\author{
        Luis Thomas\\
        \href{mailto:luis.thomas2005@gmail.com}{luis.thomas2005@gmail.com}\\
            \and
        Malo Poles\\
        \href{mailto:poles.malo@gmail.com }{poles.malo@gmail.com }\\
}

\title{SEL\\Rapport de TP\\ISTIC - Université de Rennes 1\\}

\date{Vendredi 7 Décembre 2018}

\begin{document}
\maketitle

\newpage

\section{Introduction}

Ce document témoigne du travail fourni afin de remplir les objectifs de ce TP, commençons par énumérer ces derniers.

\paragraph{Objectifs :}

L'objectif principal était de remplacer dynamiquement l’exécution d'une fonction donné dans un processus, par une autre suite d'instructions.\\
Il fallait pour cela réussir à arrenter le processus, puis allouer de la mémoire afin de copier les nouvelles instructions, 
puis enfin obliger le processus a exécuter ces instructions plutôt que celle de la fonction donné en utilisant un trampoline.\\

Le code source est disponible sur GitLab a cette adresse : \url{https://gitlab.com/Luisky/sel_tp}


\section{Travail Accompli}

Afin de réaliser ces objectifs nous avons utilisé la fonction ptrace() de la libc, utilisant elle même l'appel système sys\_ptrace (101 sur x86\_64).\\

Cette fonction permet a un processus d'interagir avec un autre processus, en modifiant par exemple l'état de sa mémoire ou de ses registres.

Ptrace permet a un processus de se laisser tracer par un autre processus (PTRACE\_TRACEME), nous n'avons pas utilisé cette fonctionnalité car il
s'agissait de modifier la mémoire d'un processus sans son accord préalable.

\section{Résultats}
In this section we describe the results.

\section{Conclusions}
We worked hard, and achieved very little.

\end{document}





